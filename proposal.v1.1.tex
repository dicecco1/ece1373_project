\documentclass[conference,compsoc]{IEEEtran/IEEEtran}
% Some/most Computer Society conferences require the compsoc mode option,
% but others may want the standard conference format.
%
% If IEEEtran.cls has not been installed into the LaTeX system files,
% manually specify the path to it like:
% \documentclass[conference,compsoc]{../sty/IEEEtran}


%\documentclass{article}

\sloppy
\usepackage{geometry}
\geometry{letterpaper, margin=1in}
\usepackage[utf8]{inputenc}
\usepackage{graphicx}
%\usepackage[breaklinks=true]{hyperref}
\usepackage{hyperref}
\usepackage[T1]{fontenc}

\title{\bf Final Project Proposal}
\author{ECE 1373 Winter 2016}
\date{\today}

\begin{document}

\maketitle

\section{Group Members}
Joe Smith, John Doe, Jane Doe

\section{Introduction}
\begin{itemize}
\item Overview of what you want to build.
\item A brief description why this project is interesting and why you want to do it.
\item A brief literature search to identify similar projects and then investigating the pros and cons of them.
\end{itemize}

\section{Goal}
\begin{itemize}
\item 
A brief explanation of what you want to achieve in your project and if possible what
advantage it has over the existing methods.
\end{itemize}

\section{Specifications}
Write this section in point form.
\begin{itemize}
\item Which board are you considering to use?  Explain why.
\item Functions and features of your design.
Describe any algorithms you want to implement.
\item Peripheral Requirements: other equipment or external hardware needed (e.x. camera, VGA display, keyboard, etc..).
\item Constraints and limitations of your design.
\end{itemize}

\section{System Overview}
\begin{itemize}
\item Provide an adequate description of the system you want to build for your design.
\item What will be implemented using high-level synthesis?
\item What high-level synthesis flow do you plan to use?
\item You may include a flow chart (e.x. to describe the flow of data between various functions of your algorithm)
\item Include a block diagram and refer to it.
An example block diagram can be viewed at \href{http://www.eecg.toronto.edu/~pc/courses/532/2008/handouts/blockdiagram.pdf}{http://www.eecg.toronto.edu/\~{}pc/courses/532 /2008/handouts/blockdiagram.pdf}.
Note that it should also indicate the blocks you need to custom-build and the ones that you are obtaining from elsewhere.
\end{itemize}


\section{Testing}
\begin{itemize}
\item 
Include your testing procedure as well as test cases.
For example, describe how you would test key features.
\end{itemize}


\section{Risk}
\begin{itemize}
\item Identify risks attributed to your design which may cause your project to fail.
\item Provide a contingency plan, develop milestones accordingly -- see Milestones section, for example, plan a catch up week where you are not planning any deliverables.
\item Strengths and weaknesses of your team.
\end{itemize}

 
\section{Milestones}
\begin{itemize}
\item Point form list of goals you plan to achieve each week.
\item Describe the responsibilities of each group member.
\item It is strongly recommended that you develop several working phases in your project, each of which could be used as a final demo.  Unexpected things happen, bugs that cannot be found, unexpected complexity... The worst case scenario is to have nothing working at the end.  Plan for intermediate milestones that can be demonstrated.
\end{itemize}

  \end{document}
  